%============== PREAMBOLO E DICHIARAZIONI INIZIALI ==========
\documentclass[10pt,oneside,a4paper]{article}

\usepackage[latin1]{inputenc} 
\usepackage[italian]{babel}
\usepackage{siunitx} %Inserisce automaticamente i dati con le unit� di misura correttamente formattate del SI (utilizzo: \SI{0.82}{m^2}, in generale \SI{misura con il punto decimale}{unit� di misura})
\sisetup{output-decimal-marker = {,}, separate-uncertainty = true, input-uncertainty-signs = \pm, detect-weight=true, detect-family=true} %per usare SI con la virgola decimale
\usepackage{listings} %Per citare codice informatico formattandolo correttamente
\usepackage{amsmath}
\usepackage{graphicx}
\usepackage{epigraph}
\usepackage{booktabs}	%tabelle migliorate
\usepackage{tablefootnote}	%note a pi� di pagina in tabella
\usepackage{threeparttable} %tabella con note a pi� di tabella
\usepackage{caption}	%descrizione per figure
\captionsetup{tableposition=top,figureposition=bottom,font=small} %setup descrizione
\usepackage{float}
\usepackage{esvect} %vettori

\setcounter{section}{-1}

%========= PRIMA PAGINA ===========
\title{\textsc{Misura della costante elastica di una molla e dell'accelerazione di gravit� }}
\author{\small{G. Galbato Muscio} \and \small{L. Gravina} \and \small{L. Graziotto} \and \small{M. Rescigno}}
\date{}

\begin{document}
\begin{figure}
	\centering
	\includegraphics[scale=0.5,trim={2.8cm 8.9cm 0 9cm},clip]{logo.png}
\end{figure}
\maketitle
\begin{center} 
\fbox{{\fontsize{13pt}{8mm}\textsc{Gruppo B2.3}}} \\
\vspace{1cm}
\begin{tabular}{ccc}
	 Esperienza di laboratorio && Consegna della relazione \\
	  \emph{\small{6 aprile 2017}} && \emph{\small{11 aprile 2017}} \\
\end{tabular} 

\vspace{0.5cm}

\end{center}
\hrule
\vspace{0.5cm}
\begin{abstract}
L'accelerazione di gravit� $g$ influenza il moto oscillatorio di una massa appesa ad una molla. Studiando il periodo e l'allungamento di essa, ne calcoliamo la costante elastica $k$ e stimiamo $g$.
\end{abstract}
\vspace{0.5cm}
\newpage
\tableofcontents %Indice
\listoftables %Indice delle tabelle
\listoffigures %Indice dei grafici
\pagebreak
\section{Convenzioni e formule}
In questa relazione verranno usate le seguenti convenzioni:
\begin{enumerate}
	\item sar� usata la virgola [ $,$ ] come separatore decimale;
	\item l'approssimazione decimale della cifra $5$ sar� fatta per eccesso;
	\item al fine di migliorare la qualit� dell'elaborazione dei dati, ogni grafico/istogramma prodotto a mano su carta millimetrata sar� riportato insieme al suo equivalente prodotto attraverso un software di analisi dati\footnote{In questo contesto i dati sono stati elaborati con il software di analisi \emph{R}.};
	\item al fine di snellire la relazione e migliorarne la leggibilit�, riporteremo nel corpo del documento solamente le tabelle riepilogative e dedicheremo un'appendice finale alle tabelle contenenti tutte le singole misure e i singoli risultati. 
\end{enumerate}
Inoltre, si far� riferimento alle seguenti formule:
\begin{enumerate}
	\item media 
		\begin{equation}\label{eq:media}
			\bar{x} = \frac{1}{N}\sum_{i=1}^Nx_i;
		\end{equation}
	\item varianza
		\begin{equation}\label{eq:varianza}
			\sigma^2 = \frac{1}{N}\sum_{i=1}^N(x_i-\bar{x})^2;
		\end{equation}
	\item deviazione standard
		\begin{equation}\label{eq:deviazione}
			\sigma = \sqrt{\sigma^2}.
		\end{equation}	
\end{enumerate}

%===============SCOPO E DESCRIZIONE DELL'ESPERIENZA==============
\section{Scopo e descrizione dell'esperienza}
\label{sec:description}
Una molla di costante elastica $k$ a cui � attaccata una massa $m$ soggetta alla forza peso $m\vec{g}$, reagisce con una forza data dalla \textbf{Legge di Hooke} $F=-k(x-x_0)$, e si porta nella posizione di equilibrio $x_{\text{eq}}=x_0+mg/k$; spostando la massa dalla posizione di equilibrio, si origina un moto armonico di periodo $T=2\pi\sqrt{m/k}$.

In questa esperienza calcoliamo in modo indiretto la costante elastica $k$ della molla a partire dalle misure del periodo di oscillazione, e misurando la posizione di equilibrio in funzione della massa applicata, stimiamo la costante di accelerazione gravitazionale $g$. 

Adottiamo due metodi diversi:
\begin{enumerate}
\item misuriamo ripetutamente periodo e allungamento e dai valori medi calcoliamo $k$ e $g$;
\item misuriamo il periodo e l'allungamento in funzione della massa applicata e, graficamente, ricaviamo i coefficienti di proporzionalit� tra determinati valori che ci consentono di estrarre $k$ e $g$.
\end{enumerate}

%================APPARATO SPERIMENTALE======================
\section{Apparato Sperimentale}

\subsection{Strumenti}
\label{subsec:strumenti}
\begin{itemize}
\item Molla appesa ad un supporto con carta millimetrata per misurarne l'allungamento [divisione \SI{1}{mm}, incertezza \SI{0,3}{mm}];
\item Bilancia per la misura della massa dei campioni [risoluzione \SI{0,1}{g}, incertezza \SI{0,03}{g}, portata \SI{2000}{g}];
\item Cronometro a lettura digitale per le misure di periodo [risoluzione \SI{0,01}{s}, incertezza \SI{0,003}{s}];
\item Squadra per ridurre l'errore di parallasse nella misura di allungamento.
\end{itemize}

\subsection{Campioni}
\label{subsec:campioni}
\begin{itemize}
\item $10$ dischetti che si possono appendere alla molla.
\end{itemize}

%==============SEQUENZA OPERAZIONI SPERIMENTALI============
\section{Sequenza Operazioni Sperimentali}

\subsection{Verifica degli strumenti}
\label{subsec:verifica}
Per quanto riguarda la molla, notiamo che applicando meno di tre dischetti questa non si deforma, dunque non possiamo compiere misure di allungamento o di periodo in tale circostanza (il problema sar� meglio trattato nel paragrafo~\ref{subsec:metodo2}). Inoltre scegliamo di adottare un'incertezza di \SI{0,3}{mm} per l'allungamento in quanto non riusciamo a interpolare tra meno di mezza tacca: la nostra risoluzione � dunque \SI{0,5}{mm} e l'incertezza � pari a $\SI{0,5}{mm}/\sqrt{3}=\SI{0,3}{mm}$. La bilancia pu� essere tarata prima di ogni misurazione e assumiamo come incertezza $0,3$ volte la sua risoluzione. La misura del periodo non � compromessa dal tempo di reazione dello sperimentatore nell'azionare il cronometro in quanto stimiamo che l'intervallo tra l'inizio del fenomeno e la partenza del cronometro sia pari a quello tra la fine del fenomeno e lo stop del cronometro.

%============== METODO 1 ================================
\subsection{Metodo 1}
\label{subsec:metodo1}
Misuriamo la massa complessiva di $5$ dischetti e quindi di tutti i $10$ dischetti con la bilancia, e quindi l'allungamento della molla a cui essi sono applicati, ripetendo in entrambi i casi le misure $5$ volte. Eseguiamo poi $50$ misure ripetute di $5$ periodi di oscillazione e successivamente $5$ misure ripetute di $50$ periodi di oscillazione, applicando sia $5$ dischetti sia $10$ dischetti.

I dati sperimentali sono riportati nelle tabelle~\ref{tab:metodo1_allungamento-massa},\ref{tab:5periodi_5dischi},\ref{tab:5periodi_10dischi},\ref{tab:50periodi_5dischi} e \ref{tab:50periodi_10dischi}, e negli istogrammi di figure~\ref{fig:ist_5dischi} e \ref{fig:ist_10dischi}.


%============= METODO 2 =================================
\subsection{Metodo 2}
\label{subsec:metodo2}
Misuriamo in modo integrale la massa dei dischetti e dunque l'allungamento della molla aggiungendoli progressivamente. Eseguiamo al contempo per ogni campione aggiunto $20$ misure ripetute del tempo impiegato per compiere $10$ oscillazioni. I dati sperimentali raccolti sono riportati nelle tabelle~\ref{tab:metodo2_allungamento-massa} e \ref{tab:metodo2_periodo}.

Tracciamo quindi il grafico (figura~\ref{fig:periodoquadro-massa}) di $T^2$ in funzione della massa $m$, e quello (figura~\ref{fig:xeq-massa}) dell'allungamento in funzione della massa $m$. Estraiamo i coefficienti angolari delle rette che meglio approssimano i punti sperimentali e ricaviamo $k$ e $g$.


%============= CONSIDERAZIONI FINALI ====================
\section{Considerazioni finali}
\label{sec:c_finali}

%============= TABELLE e GRAFICI =======================
\newpage
\section{Appendice: tabelle e grafici}







\end{document}
